\documentclass[a5paper, 20pt, titlepage]{article}
\def\MakeUppercaseUnsupportedInPdfStrings{\scshape}
\usepackage[warn]{mathtext}
\usepackage{cmap}
\usepackage[T2A]{fontenc}
\usepackage[utf8]{inputenc}
\usepackage[russian]{babel}
\usepackage{amsmath}
\usepackage[ warn ]{ mathtext }
\usepackage{amsfonts}
\usepackage{amssymb}
\usepackage[normalem]{ulem}
\usepackage[pdftex]{graphics}
\usepackage{graphicx}
\usepackage{wrapfig}
\usepackage{amsmath,systeme}
\usepackage{comment}
\usepackage{slashbox}
\usepackage{pgfplots}
\usepackage{setspace}
\usepackage{geometry}
\usepackage[unicode, pdftex]{hyperref}
\usepackage{yfonts}
\usepackage{mathtools}
\newtheorem{theorem}{Теорема}
\pgfplotsset{compat=newest}
\usepgfplotslibrary{fillbetween}
\geometry{verbose,a4paper,tmargin=2cm,bmargin=2cm,lmargin=2.5cm,rmargin=1.5cm}
\setcounter{MaxMatrixCols}{25}
\hypersetup{
    colorlinks = true,
    linkbordercolor = {white},
    linkcolor=blue
}

\setstretch{1.4}

\title{Самостійна робота 2 з функціонального аналізу \\ Варіант 4}
\author{\vspace{-4mm} КА-02 Козак Назар}
\date{}

%комани для векторів x та y
\newcommand{\x}{\textbf{x}}
\newcommand{\y}{\textbf{y}}

\newcommand{\Int}[1]{\int \limits_{0}^{1} #1 ds}
\begin{document}

\maketitle


\paragraph{Задача 1.} \hfill \nolinebreak З'ясувати чи буде банаховим лінійний нормований простір $\left( E, \| \cdot \| \right)$, якщо \newline
$E = \left\{ f \in \mathcal{C}\left([0, 1] \right); \mathbb{C} : f(0)  = e^{-i \frac{\pi}{4}} f(1) \right\}, \| f \| = \| f \|_{\infty}$.

\noindent\rule{4cm}{0.4pt}

Спочатку нагадаємо твердження з лекцій:

\vspace{4mm}
\textbf{Тверждення.} \quad Нехай $\left( E, \| \cdot \| \right)$ -- банаховий простір, а $E_0$ -- замкнутий лiнiйний пiдпростiр простору $E$.
Тоді $\left( E_0, \| \cdot \| \right)$ -- банаховий простір. 
\vspace{4mm}

З нього випливає, що щоб довести, що  $\left( E, \| \cdot \|_\infty \right)$ - банаховий простір нам достатньо довести, що $E$ -- замкнутий лiнiйний пiдпростiр простору $\bar E$, такого що $\left(\bar E, \| \cdot \|_\infty \right)$ -- банаховий простір. Тому розділимо доведення на дві частини:

\begin{enumerate}
\item Як відомо з лекцій \hyperlink{1}{[1]}: лінійний нормований простір $\mathcal{C}_b \left( \Omega, \mathbb{K} \right)$ є банаховим, коли $\Omega \subset \mathbb{R}^n$. У нашому випадку $\Omega = [0, 1] \subset \mathbb{R}$, тому $\left( \mathcal{C}_b  \left( [0, 1], \mathbb{C} \right), \| \cdot \|_\infty \right)$ є банаховим простором.

З іншого боку $f \in E \Rightarrow{} f \in C([0,1], \mathbb{C})$. Тобто кожна функція з лінійного простору $E$ є неперервною на відрізку $[0, 1]$ з чого за теоремою Вейєрштрасса випливає те, що вона є обмеженою на $[0, 1]$. Отже, $E \subset \mathcal{C}_b  \left( [0, 1], \mathbb{C} \right)$. Тепер доведемо, що $E$ є лінійним підпростором простору $\mathcal{C}_b  \left( [0, 1], \mathbb{C} \right)$, для цього достатньо перевірити дві умови:

\begin{enumerate}
\item Нехай $\textbf{x}, \textbf{y} \in E$, тобто  $\textbf{x}(0) = e^{-i \frac{\pi}{4}} \textbf{x}(1), \quad \textbf{y}(0) = e^{-i \frac{\pi}{4}} \textbf{y}(1)$ , тоді

\vspace{3mm}

$(\x + \y)(0) = \x(0) + \y(0) = e^{-i \frac{\pi}{4}} \x(1) +  e^{-i \frac{\pi}{4}} \y(1) =  e^{-i \frac{\pi}{4}}(\x(1) + \y(1)) =  e^{-i \frac{\pi}{4}}(\x + \y)(1) 
\Rightarrow{} \quad \x + \y \in E$

\vspace{4mm}
\item Нехай $\x \in E$, $\lambda \in \mathbb{C}$, тоді

\vspace{3mm}
 
$(\lambda \x)(0) = \lambda \x(0) = \lambda  e^{-i \frac{\pi}{4}} \x(1) = e^{-i \frac{\pi}{4}} (\lambda \x(1)) = e^{-i \frac{\pi}{4}} (\lambda \x)(1) \quad \Rightarrow \quad (\lambda \x) \in E $
\end{enumerate}

Отже, маємо: $E$ є лінійним підпростором простору $\mathcal{C}_b  \left( [0, 1], \mathbb{C} \right)$, такого що $\left( \mathcal{C}_b  \left( [0, 1], \mathbb{C} \right), \| \cdot \|_\infty \right)$ є банаховим простом.
 

\item Тепер доведемо замкненість $E$:

Нехай $f^* \in E'$. З цього випливає, що $\exists \left( f_n\right)_{n = 1}^{\infty} \in E: \lim \limits_{n \to \infty} f_n = f^* \text{відносно $\| \cdot \|_\infty$}$.

\vspace{-2mm}
Також маємо що $f_n \in E \Leftrightarrow{} f_n(0) = f_n(1) e^{-i \frac{\pi}{4}}$. Тоді:
\vspace{-2mm}
$$\lim \limits_{n \to \infty} f_n = f^* \text{відносно $\| \cdot \|_\infty$} \quad \Leftrightarrow{} \quad \forall \varepsilon >0: \exists N \in \mathbb{N}: \forall n \geqslant N: \| f_n - f^* \|_\infty < \varepsilon \quad \Leftrightarrow $$
\vspace{-6mm}
$$\Leftrightarrow{} \quad \forall \varepsilon >0: \exists N \in \mathbb{N}: \forall n \geqslant N: \sup \limits_{t \in [0, 1]}|f_n - f^* | < \varepsilon \quad
\Leftrightarrow{} \quad  \forall t \in [0, 1]: \forall \varepsilon >0: \exists N \in \mathbb{N}: \forall n \geqslant N: |f_n - f^* | < \varepsilon \,\, (*)$$

Оскільки остання умова виконується для  $\forall t \in [0, 1]$, то розглянемо випадок, коли $t = 0$:
\vspace{-3mm}
$$\forall \varepsilon >0: \exists N \in \mathbb{N}: \forall n \geqslant N: |f_n(0) - f^*(0)| < \varepsilon \quad \Leftrightarrow{} \quad
\forall \varepsilon >0: \exists N \in \mathbb{N}: \forall n \geqslant N: |f_n(1) e^{-i \frac{\pi}{4}} - f^*(0)| < \varepsilon \Leftrightarrow{}$$
\vspace{-8mm}
$$\Leftrightarrow{} \lim \limits_{n \to \infty} (f_n(1) e^{-i \frac{\pi}{4}}) = f^*(0) \quad \Leftrightarrow{} \quad e^{-i \frac{\pi}{4}} \lim \limits_{n \to \infty} f_n(1) = f^*(0) \quad \Leftrightarrow{} \quad  \lim \limits_{n \to \infty} f_n(1) =e^{i \frac{\pi}{4}}  f^*(0)  $$

\newpage{}

Тепер доведемо, що $\lim \limits_{n \to \infty} f_n(1) = f^*(1)$, для цього в $(*)$ візьмемо $t = 1 \in [0, 1]$. Маємо:
\vspace{-3mm}
$$\forall \varepsilon >0: \exists N \in \mathbb{N}: \forall n \geqslant N: |f_n(1) - f^*(1)| < \varepsilon \quad \Leftrightarrow{} \quad \lim \limits_{n \to \infty} f_n(1) = f^*(1)$$
\vspace{-3mm}
Отже, маємо:
\vspace{2mm}
$$ \lim \limits_{n \to \infty} f_n(1) = e^{i \frac{\pi}{4}}  f^*(0) \quad \Leftrightarrow{} \quad  f^*(1) = e^{i \frac{\pi}{4}}  f^*(0)  \quad \Leftrightarrow{} \quad f^*(0) =  e^{-i \frac{\pi}{4}} f^*(1) \quad \Leftrightarrow{} \quad f^* \in E \quad \Leftrightarrow{}$$
\vspace{-8mm}
$$ \Leftrightarrow{} \quad E \,\,\, \text{є замкнутим}$$
\end{enumerate}

Остаточно маємо $E$ -- замкнутий лінійний підпростір простору $ \mathcal{C}_b  \left( [0, 1], \mathbb{C} \right)$, такого, що $\left( \mathcal{C}_b  \left( [0, 1], \mathbb{C} \right), \| \cdot \|_\infty \right)$ є банаховим простором, а, отже, $\left( E, \| \cdot \|_\infty \right)$ є банаховим простором.

\paragraph{Відповідь.}  Так,  $\left( E, \| \cdot \|_\infty \right)$ є банаховим простором.

\vspace{4mm}


\paragraph{Задача 2.} Для вiдображення $\phi: \mathcal{C} \left( [0, 1]; \mathbb{R} \right) \rightarrow \mathcal{C} \left( [0, 1]; \mathbb{R} \right)$
знайти а) значення $\lambda$, для яких $\phi$ є стиском; б) знайти розв'язок рівняння $f = \phi(f)$ для одного із знайдених значень $\lambda \neq 0$, якщо: \newline $\phi(f) = \lambda \int \limits_{0}^{1} (t + s)^2 f(s) ds - 2t^2$.

\noindent\rule{4cm}{0.4pt}

\begin{enumerate}

\item Спочатку нагадаємо означення стиску:

\textbf{Означення.} \quad Нехай $\left( E, \| \cdot \| \right)$ - лінійний нормований простір та $X \subset E$. Відображення $\varphi: X \rightarrow X$ називається стиском на $X$, якщо існує таке $q \in (0, 1)$, що 
\vspace{-3mm}
$$ \| \varphi(x') - \varphi(x'') \| \leqslant q \| x' - x'' \|, \qquad \text{для всіх $x', x'' \in X$.}$$

Нехай $f, g \in  \mathcal{C} \left( [0, 1]; \mathbb{R} \right)$. Оскільки даний лінійний простір - простір неперервних функцій, тоді розглядаємо супремум-норму, тобто $\| f \|_\infty = \sup \limits_{t \in [0, 1]} |f|$. Розглянемо вираз $\| \phi(f) - \phi(g)\|_\infty$:

%---------
$\| \phi(f) - \phi(g)\|_\infty = \left\|  \lambda \int \limits_{0}^{1} (t + s)^2 f(s) ds - 2t^2 -  \lambda \int \limits_{0}^{1} (t + s)^2 g(s) ds + 2t^2 \right\|_\infty = \left\| \lambda  \int \limits_{0}^{1} (t + s)^2 \left(f(s) - g(s) \right) ds  \right\|_\infty = $

\vspace{3mm}
\hspace{23mm}
$= \sup \limits_{t \in [0, 1]} \left| \lambda  \int \limits_{0}^{1} (t + s)^2 \left(f(s) - g(s) \right) ds \right|
= |\lambda| \sup \limits_{t \in [0, 1]} \left| \int \limits_{0}^{1}  (t + s)^2 \left(f(s) - g(s) \right)  ds \right| \leqslant $

\vspace{3mm}
\hspace{23mm}
$\leqslant |\lambda| \sup \limits_{t \in [0, 1]}  \int \limits_{0}^{1} (t + s)^2 \left| \left(f(s) - g(s) \right) \right| ds \leqslant
|\lambda| \sup \limits_{t \in [0, 1]}  \int \limits_{0}^{1}  (t + s)^2  \left\| \left(f(s) - g(s) \right) \right\|_\infty ds =$

\vspace{3mm}
\hspace{23mm}
$= |\lambda| \cdot \left\| \left(f(s) - g(s) \right) \right\|_\infty  \sup \limits_{t \in [0, 1]}  \int \limits_{0}^{1} (t + s)^2 ds =
|\lambda| \cdot \left\| \left(f(s) - g(s) \right) \right\|_\infty  \sup \limits_{t \in [0, 1]}  \left( \left. \frac{(t+s)^3}{3} \right|_{s = 0}^{s = 1} \right) =$

\vspace{3mm}
\hspace{23mm}
$= |\lambda| \cdot \left\| \left(f(s) - g(s) \right) \right\|_\infty  \sup \limits_{t \in [0, 1]}  \left( \frac{(t+1)^3}{3} - \frac{t^3}{3} \right) = 
|\lambda| \cdot \left\| \left(f(s) - g(s) \right) \right\|_\infty  \sup \limits_{t \in [0, 1]}  \left(\frac{3t^2+3t + 1}{3} \right) =$

\vspace{3mm}
\hspace{23mm}
$= |\lambda| \cdot \left\| \left(f(s) - g(s) \right) \right\|_\infty \cdot \frac{7}{3} = \frac{7}{3} \cdot |\lambda| \cdot \left\| \left(f(s) - g(s) \right) \right\|_\infty$

%-----
\newpage{}
Отже, ми отримали:

\vspace{-4mm}
$$ \| \phi(f) - \phi(g)\|_\infty \leqslant  \frac{7}{3} \cdot |\lambda| \cdot \left\| \left(f(s) - g(s) \right) \right\|_\infty$$

Згідно з означення $\phi$ буде стиском тоді, коли $ \frac{7}{3} \cdot |\lambda| \in (0, 1)$. Тоді маємо:

\vspace{-3mm}
$$\frac{7}{3} \cdot |\lambda| \in (0, 1) \quad \Leftrightarrow \quad |\lambda| \in (0, \frac{3}{7} ) \quad \Leftrightarrow \quad
\lambda \in (-\frac{3}{7}, 0) \cup (0, \frac{3}{7}) $$

Отже, ми отримали: якщо $\lambda \in (-\frac{3}{7}, 0) \cup (0, \frac{3}{7})$, то $\phi$ є стиском на $\mathcal{C} \left( [0, 1]; \mathbb{R} \right)$.

\item Як відомо з лекцій \hyperlink{1}{[2]}: коли $\Omega \subset \mathbb{R}^n$ є компактною множиною -- лінійний нормований простір $\mathcal{C} \left( \Omega; \mathbb{K} \right)$ є банаховим. В нашому випадку $\Omega = [0, 1] \subset \mathbb{R}$ - обмежена та замкнена множина, тому вона є компактною, а, отже, $\left( \mathcal{C} \left( [0, 1], \mathbb{R} \right), \| \cdot \|_\infty \right)$ - банаховий простір. З цього за теоремою Банаха маємо те, що стиск $\phi$ на $\mathcal{C} \left( [0, 1], \mathbb{R} \right)$ має єдину нерухому точку, тобто існує тільки один розв'язок даного рівняння: $f = \phi(f)$ в $\mathcal{C} \left( [0, 1], \mathbb{R} \right)$.

Візьмемо $\lambda = \frac{1}{4} \in  (-\frac{3}{7}, 0) \cup (0, \frac{3}{7})$. Тоді маємо:

\vspace{3mm}
$ f =  \frac{1}{4} \int \limits_{0}^{1} (t + s)^2 f(s) ds - 2t^2 = \frac{1}{4} \int \limits_{0}^{1} (t^2 + 2st + s^2) f(s) ds - 2t^2 = \frac{1}{4} \int \limits_{0}^{1} t^2 f(s) ds + \frac{1}{4} \int \limits_{0}^{1} 2stf(s) ds + \frac{1}{4} \int \limits_{0}^{1} s^2 f(s) ds - \nolinebreak 2t^2 =$

\vspace{3mm}
\hspace{2mm} 
$=  \frac{1}{4} t^2 \int \limits_{0}^{1} f(s) ds + \frac{1}{2}t \int \limits_{0}^{1} sf(s) ds + \frac{1}{4} \int \limits_{0}^{1} s^2 f(s) ds - \nolinebreak 2t^2 =
t^2 \left( \frac{1}{4} \int \limits_{0}^{1} f(s) ds - 2 \right) +  \frac{1}{2}t \int \limits_{0}^{1} sf(s) ds + \frac{1}{4} \int \limits_{0}^{1} s^2 f(s) ds  $

\vspace{3mm}
Покладемо:

$$ 
\begin{cases}
A = \frac{1}{4} \int \limits_{0}^{1} f(s) ds - 2 \\ 
B = \frac{1}{2} \int \limits_{0}^{1} sf(s) ds \\
C = \frac{1}{4} \int \limits_{0}^{1} s^2 f(s) ds
\end{cases}
$$

\vspace{3mm}
Бачимо, що $A,B,C \in \mathbb{R}$. Отримали:
$$ f(t) = At^2 + Bt + C$$

Виразимо кожну з констант окремо:

\begin{enumerate}
\item A: 

$A = \frac{1}{4} \int \limits_{0}^{1} f(s) ds - 2 = \frac{1}{4} \int \limits_{0}^{1} (As^2 + Bs + C) ds - 2 = \frac{1}{4} A \int \limits_{0}^{1} s^2 ds + \frac{1}{4} B \int \limits_{0}^{1} s ds + C \int_{0}^{1} ds - 2 = $

\vspace{3mm}
\hspace{2mm}
$= \frac{1}{4} A \left. \frac{s^3}{3} \right|_{s = 0}^{s = 1} + \frac{1}{4} B \left. \frac{s^2}{2} \right|_{s = 0}^{s = 1} + \frac{1}{4}C \left. s \right|_{s = 0}^{s = 1} - 2 = \frac{1}{12} A + \frac{1}{8} B  + \frac{1}{4}C  - 2$

\vspace{3mm}
\item B:

$B =  \frac{1}{2} \int \limits_{0}^{1} sf(s) ds =  \frac{1}{2} \int \limits_{0}^{1} s(As^2 + Bs + C) ds = \frac{1}{2} \int \limits_{0}^{1} (As^3 + Bs^2 + Cs) ds = \frac{1}{2} A \Int{s^3} + \frac{1}{2} B \Int{s^2} + \nolinebreak \frac{1}{2} C \Int{s} =$

\vspace{3mm}
\hspace{2mm}
$= \frac {1}{2} A \left. \frac{s^4}{4}\right|_{s = 0}^{s = 1} + \frac{1}{2} B \left. \frac{s^3}{3} \right|_{s = 0}^{s = 1} + \frac{1}{2} C \left. \frac{s^2}{2}\right|_{s = 0}^{s = 1} = \frac{1}{8} A + \frac{1}{6} B + \frac{1}{4} C$

\newpage

\item C:

$C = \frac{1}{4} \int \limits_{0}^{1} s^2 f(s) ds = C = \frac{1}{4} \int \limits_{0}^{1} s^2 (As^2 + Bs + C) ds = \frac{1}{4} \Int{(As^4 + Bs^3 + Cs^2)} = $

\vspace{3mm}
\hspace{2mm}
$= \frac{1}{4} A \Int{s^4} + + \frac{1}{4} B\Int{s^3} + \frac{1}{4} C \Int{s^2} = \frac{1}{4} A \left. \frac{s^5}{5} \right|_{s = 0}^{s = 1} + \frac{1}{4} A \left. \frac{s^4}{4} \right|_{s = 0}^{s = 1} + \frac{1}{4} C \left. \frac{s^3}{3} \right|_{s = 0}^{s = 1} =$

\vspace{3mm}
\hspace{2mm}
$= \frac{1}{20} A + \frac{1}{16} B + \frac{1}{12} C$
\end{enumerate}

Отримали систему:

$$
\begin{cases}
A = \frac{1}{12} A + \frac{1}{8} B  + \frac{1}{4}C  - 2 \\
B = \frac{1}{8} A + \frac{1}{6} B + \frac{1}{4} C \\
C = \frac{1}{20} A + \frac{1}{16} B + \frac{1}{12} C
\end{cases}
\quad \Leftrightarrow \quad 
\begin{cases}
\frac{11}{12} - \frac{1}{8} B  - \frac{1}{4}C  = 2 \\
\frac{1}{8} A - \frac{5}{6} B + \frac{1}{4} C = 0 \\
\frac{1}{20} A + \frac{1}{16} B - \frac{11}{12} C = 0
\end{cases}
$$

Розв'яжемо систему методом Крамера:

$$\Delta = 
\begin{vmatrix}
\frac{11}{12} & -\frac{1}{8} & -\frac{1}{4} \\
\frac{1}{8}    & -\frac{5}{6} & \frac{1}{4}  \\
\frac{1}{20}  & \frac{1}{16} & -\frac{11}{12} \\  
\end{vmatrix}
=\cfrac{45457}{69120}; 
\qquad
\Delta_A =  
\begin{vmatrix}
2 & -\frac{1}{8} & -\frac{1}{4} \\
0    & -\frac{5}{6} & \frac{1}{4}  \\
0  & \frac{1}{16} & -\frac{11}{12} \\  
\end{vmatrix}
= \cfrac{431}{288}
\qquad
\Delta_B =  
\begin{vmatrix}
\frac{11}{12} & 2 & -\frac{1}{4} \\
\frac{1}{8}    & 0 & \frac{1}{4}  \\
\frac{1}{20}  & 0 & -\frac{11}{12} \\  
\end{vmatrix}
= \cfrac{61}{240}
$$

\vspace{-1mm}
$$
\quad
\Delta_C =  
\begin{vmatrix}
\frac{11}{12} & -\frac{1}{8} & 2 \\
\frac{1}{8}    & -\frac{5}{6} & 0  \\
\frac{1}{20}  & \frac{1}{16} & 0 \\
\end{vmatrix}
= \cfrac{19}{192}
$$

Тоді:

$$
\begin{cases}
A = \frac{\Delta_A}{\Delta} = \frac{45457}{69120} \cdot \frac{288}{431} \\
B = \frac{\Delta_B}{\Delta} = \frac{45457}{69120} \cdot \frac{240}{61}\\
C = \frac{\Delta_C}{\Delta} = \frac{45457}{69120} \cdot \frac{192}{19}
\end{cases}
\quad \Rightarrow \quad 
\begin{cases}
A = \frac{45457}{103440}  \\ 
B = \frac{45457}{17568} \\
C = \frac{45457}{6840}
\end{cases}
$$

\vspace{3mm}
Отже, отримуємо функцію $f$, яка і буде єдиним розв'язком заданого рівняння:

$$f(t) = \frac{45457}{103440} t^2 +  \frac{45457}{17568} t + \frac{45457}{6840}$$ 
\end{enumerate}

\paragraph{Відповідь.} а)  $\lambda \in (-\frac{3}{7}, 0) \cup (0, \frac{3}{7})$ 

\vspace{1mm}
\hspace{16mm}
б) $f(t) = \frac{45457}{103440} t^2 +  \frac{45457}{17568} t + \frac{45457}{6840}$.

\newpage{}

\paragraph{Задача 3.} Встановити чи буде множина $X$ предкомпактною в лінійному нормованому просторі $\left(E, \| \cdot \| \right)$ якщо $X = \left\{  \x = (x_1, x_2, \dots) \in \ell_1: \left| x_n \right| \leqslant \frac{1}{n^2}\right\}, E = \ell_1$. 

\noindent\rule{4cm}{0.4pt}

Нагадаємо теорему з лекцій:

\textbf{Теорема.} \quad  Нехай $p \in [1, +\infty)$. Підмножина $F \subset \ell_p$ є предкомпактною тоді і тільки тоді, коли
\begin{enumerate}
\item підмножина $F$ є обмеженою за нормою простору $\ell_p$; \vspace{-3mm}
\item для довільного $\varepsilon > 0$ існує таке $n \in \mathbb{N}$, що для всіх $\x \in F, \x = (x_1, x_2, \dots)$, $$\left( \sum \limits_{k = n + 1}^{\infty}|x_k|^p\right)^{\frac{1}{p}} < \varepsilon$$
\end{enumerate}

Зазначимо, що, оскільки нам заданий лінійний нормований простір $\ell_1$, то нормою на ньому буде $\| \cdot \|_1$. Тобто для $\x \in \ell_1, \x = (x_1, x_2, \dots): \| \x \|_ 1 = \sum \limits_{k = 0}^{\infty} |x_k|$. Розділимо доведення на дві частини згідно з теоремою наведеною вище:

\begin{enumerate}
\item Нехай $\x \in X, \x = (x_1, x_2, \dots)$, тоді:

$\| \x \|_1 = \sum \limits_{k = 0}^{\infty} |x_k| \leqslant \sum \limits_{k = 1}^{\infty} \frac{1}{k^2} - \text{нерівність випливає з означення множини $X$}$ 

З курсу математичного аналізу відомо, що $\sum \limits_{k = 1}^{\infty} \frac{1}{k^2} = \frac{\pi^2}{6}$. Отже, з довільності вибору $\x$ ми отримали те, що $\forall \x \in X: \| \x \|_1 \leqslant \frac{\pi^2}{6}$, що означає обмеженість множини $X$ за нормою простору $\ell_1$.

\vspace{3mm}
\item Нехай $\x \in X, \x = (x_1, x_2, \dots)$. З умови маємо: $\forall \x \in X: |x_n| \leqslant \frac{1}{n^2}$, тому, оскільки \newline $\forall k \in \mathbb{N}: |x_k| \geqslant 0, \frac{1}{k^2} > 0$, а також оскільки ряд $\sum \limits_{k = 1}^{\infty} \frac{1}{k^2}$ - збіжний, то за першою теоремою порівняння отримуємо те, що ряд  $\sum \limits_{k = 0}^{\infty} |x_k|$ - збіжний.

\vspace{3mm}
Осільки ряд $\sum \limits_{k = 0}^{\infty} |x_k|$ - збіжний, то з курсу математичного аналізу нам відомо, що збігаються всі його залшики, при чому:
$$ \lim \limits_{n \to \infty} \sum \limits_{k = n}^{\infty} |x_k| = 0 \quad \hyperlink{1}{[3]}$$

Розпишемо останню рівність:

\vspace{3mm}
$ \lim \limits_{n \to \infty} \sum \limits_{k = n}^{\infty} |x_k| = 0 \quad \Leftrightarrow \quad \forall \varepsilon > 0: \exists N \in \mathbb{N}: \forall n \geqslant N: \left| \sum \limits_{k = n}^{\infty} |x_k| - 0 \right| < \varepsilon \quad \Leftrightarrow$

\vspace{3mm}
\hspace{29mm}
$\Leftrightarrow \quad \forall \varepsilon > 0: \exists N \in \mathbb{N}: \forall n \geqslant N: \sum \limits_{k = n}^{\infty} |x_k| < \varepsilon \quad$

\vspace{3mm}
Отже, ми отримали, що для довільного $\varepsilon > 0$ існує таке $n \in \mathbb{N}$, що для всіх $\x \in F, \x = (x_1, x_2, \dots)$: 
$$ \sum \limits_{k = n + 1}^{\infty}|x_k| < \varepsilon$$
\end{enumerate}

\newpage 

Підсумуємо:

\begin{enumerate}
\item Підмножина $X$ є обмеженою за нормою простору $\ell_1$
\item Для довільного $\varepsilon > 0$ існує таке $n \in \mathbb{N}$, що для всіх $\x \in F, \x = (x_1, x_2, \dots)$: 
$$ \sum \limits_{k = n + 1}^{\infty}|x_k| < \varepsilon$$
\end{enumerate}

Отже, за теоремою наведеною вище ми маємо те, що множина $X$ є предкомпактною в лінійному нормованому просторі $\left(\ell_1, \| \cdot \|_1 \right)$

\paragraph{Відповідь.} множина $X$ є предкомпактною в лінійному нормованому просторі $\left(\ell_1, \| \cdot \|_1 \right)$

\vspace{4mm}
\section*{Додаток }

\hypertarget{1}{[1]} - Чаповський Ю.А. - конспект лекцій з функціонального аналізу  ст. 128 Теорема 1.6.17

\hspace{-6.3mm}
[2] - Чаповський Ю.А. - конспект лекцій з функціонального аналізу  ст. 128 Наслідок 1.6.18

\hspace{-6.3mm}
[3] - \href{https://drive.google.com/file/d/1TjgKePwptgl93cQ0WRwgO9wxc3RV5srA/view?usp=share_link}{Воробйов Н.Н. - теорія рядів } ст. 34 теорема внизу сторінки

\end{document}