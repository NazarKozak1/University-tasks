\documentclass[a5paper, 20pt]{article}
\def\MakeUppercaseUnsupportedInPdfStrings{\scshape}
\usepackage[warn]{mathtext}
\usepackage{cmap}
\usepackage[T2A]{fontenc}
\usepackage[utf8]{inputenc}
\usepackage[russian]{babel}
\usepackage{amsmath}
\usepackage[ warn ]{ mathtext }
\usepackage{amsfonts}
\usepackage{amssymb}
\usepackage[normalem]{ulem}
\usepackage[pdftex]{graphics}
\usepackage{graphicx}
\usepackage{wrapfig}
\usepackage{amsmath,systeme}
\usepackage{comment}
\usepackage{slashbox}
\usepackage{pgfplots}
\usepackage{setspace}
\usepackage{geometry}
\usepackage[unicode, pdftex]{hyperref}
\usepackage{yfonts}
\newtheorem{theorem}{Теорема}
\pgfplotsset{compat=newest}
\usepgfplotslibrary{fillbetween}
\geometry{verbose,a4paper,tmargin=2cm,bmargin=2cm,lmargin=2.5cm,rmargin=1.5cm}
\setcounter{MaxMatrixCols}{25}
\hypersetup{
    colorlinks = true,
    linkbordercolor = {white},
    linkcolor=blue
}

\setstretch{1.25}

\begin{document}


\paragraph{Завдання.}


$$
\begin{matrix}
2 & 0 & 0 & 1 & 0 & 0 & 0 & 2 & 2 & 2 & 0 & 0 & 0 & 0 & 0 & 3 & 0 & 3 & 2 & 0 & 1 & 1 & 0 & 0 & 4 \\
3 & 0 & 2 & 2 & 5 & 0 & 6 & 0 & 5 & 0 & 11 & 0 & 2 & 1 & 0 & 2 & 2 & 3 & 4 & 0 & 0 & 1 & 4 & 1 &  1 \\ 
1 & 0 & 1 & 0 & 2 & 2 & 0 & 0 & 1 & 2 & 0 & 1 & 1 & 0 & 3 & 0 & 1 & 0 & 4 & 2 & 5 & 1 & 2 & 0 & 1 \\ 
1 & 0 & 0 & 1 & 4 & 1 & 1 & 3 & 0 & 2 & 7 & 3 & 0 & 1 & 0 & 2 & 1 & 0 & 2 & 0 & 0 & 0 & 2 & 0 & 2 
\end{matrix}
$$

Для конкретної реалізації вибірки виконати:

\begin{enumerate}
\item Побудувати варіаційний (дискретний або інтервальний) ряд наданої вибірки.
\item Зробити графічне зображення вибірки.
\item Побудувати емпіричну функцію розподілу.
\item Знайти незміщену оцінку математичного сподівання та дисперсії.
\item Обчислити значення вибіркової медіани, моди, асиметрії.
\item Висунути гіпотезу про розподіл, за яким отримано вибірку.
\item Знайти точкові оцінки параметрів гіпотетичного закону розподілу та перевірити їх властивості.
\item Перевірити за допомогою критерію $\chi^2$ (Пірсона) гіпотезу про розподіл з рівнем значущості  $\alpha = 0.05$.
\item Знайти довірчий інтервал для параметрів гіпотетичного закону розподілу, взяв рівень надійності $\gamma = 0.95$.
\item Висновки.
\end{enumerate}


\section{Побудувати варіаційний (дискретний або інтервальний) ряд наданої вибірки.}

Для початку відсортуємо нашу реалізацію вибірки:

$$
\begin{matrix}
0 & 0 & 0 & 0 & 0 & 0 & 0 & 0 & 0 & 0 & 0 & 0 & 0 & 0 & 0 & 0 & 0 & 0 & 0 & 0 & 0 & 0 & 0 & 0 & 0 \\
0 & 0 & 0 & 0 & 0 & 0 & 0 & 0 & 0 & 0 & 0 & 0 & 0 & 0 & 0 & 0 & 1 & 1 & 1 & 1 & 1 & 1 & 1 & 1 & 1 \\
1 & 1 & 1 & 1 & 1 & 1 & 1 & 1 & 1 & 1 & 1 & 1 & 2 & 2 & 2 & 2 & 2 & 2 & 2 & 2 & 2 & 2 & 2 & 2 & 2 \\
2 & 2 & 2 & 2 & 2 & 2 & 2 & 3 & 3 & 3 & 3 & 3 & 3 & 3 & 4 & 4 & 4 & 4 & 4 & 5 & 5 & 5 & 6 & 7 & 11
\end{matrix}
$$

Як бачимо, в вибірці досить мало унікальних значень, а саме 9, тому побудуємо дискретний варіаційний ряд:


\begin{center}
\begin{tabular}{| c | c | c | c | c | c | c | c | c | c | }
	\hline
	варіанти & 0 & 1 & 2 & 3  & 4 & 5 & 6 & 7 & 11\\ \hline
	частоти $n_i$ & 41 & 21 & 20 & 7 & 5 & 3 & 1 & 1 & 1\\ \hline
	частості $\omega_i$ & 0.41 & 0,21 & 0,2 & 0,07 & 0.05 & 0.03 & 0.01 & 0.01 & 0.01  \\ \hline
\end{tabular}
\end{center}

Зробимо перевірку, сумма всіх частот має дорівнювати обсягу вибірки(100) , а сума всіх частостей має дорівнювати одиниці:

\vspace{3mm}

$\displaystyle{\sum_{i=1}^{10} n_i = 41 + 21 + 20 + 7 + 5 + 3 + 1 + 1 + 1 = 100}$

\vspace{3mm}

$\displaystyle{\sum_{i=1}^{10} \omega_i = 0.41 + 0.21 + 0.2 + 0.07 + 0.05 + 0.03 + 0.01 + 0.01 + 0.01 = 1 }$

\newpage{}

\section{Зробити графічне зображення вибірки.}


Побудуємо за дискретним варіаційним рядом полігон відносних частот:

\vspace{4mm}
\begin{tikzpicture}
	\begin{axis}[
		axis lines = middle,
		xmin = -1 , xmax = 13,
		ymin = -0.15, ymax = 0.42,
	%
	% тіки
		ytick = {-0.2, -0.1, 0.1, 0.2, 0.3, 0.41},
		xtick = {0, 1,2,3,4,5,6,7,11},
		extra x ticks = {0},
		extra x tick labels = {\!\!\!\!\!\!\! 0},
	%
	% підпис осей	
		xlabel style={below right},
		xlabel = {$x$},
		ylabel = {$\omega$},
		width = 17 cm,	
		height =11cm,
		title  = полігон відносних частот ,
		title style={yshift= - 10 cm}
	%
	]


\addplot[color = black, mark = *, only marks, mark size = 1.25pt] coordinates {(0, 0.41) (1, 0.21) (2,0.2) (3, 0.07) (4,0.05) (5,0.03) (6, 0.01) (7,0.01) (11,0.01)};
\addplot[color = cyan, ultra thick] coordinates {(0, 0.41) (1, 0.21) (2,0.2) (3, 0.07) (4,0.05) (5,0.03) (6, 0.01) (7,0.01) (11,0.01) };
\node[above] at (1, 0.21) {$0.21$};
\node[above] at (2,0.2) {$0.2$};
\node[above] at (3, 0.07) {$0.07$};
\node[above] at (4,0.05) {$0.05$};
\node[above] at (5,0.03) {$0.03$};
\node[above] at (6, 0.01) {$0.01$};
\node[above] at (7,0.01) {$0.01$};
\node[above] at (11,0.01) {$0.01$};
	\end{axis}
\end{tikzpicture}

\vspace{4mm}

Тут числа над точками це ординати відповідних точок, тобто значення частостей.


\section{Побудувати емпіричну функцію розподілу}

За дискретним варіаційним рядом побудуємо емпіричну функцію розподілу. Для ДВР вона визначається наступним чином:

$$
F_n^*(x) =
\begin{cases}
\begin{aligned}
&0,  \qquad & x \leqslant x_1^* \\
&\omega_1^{нак} = \cfrac{n_1}{n},  &x_1^* < x \leqslant x_2^*\\
&\omega_2^{нак} = \cfrac{n_1 + n_2}{n}, &x_2^* < x \leqslant x_3^*\\
&...\\
&1 , &x > x_r^*\\
\end{aligned}
\end{cases}
$$

Отже маємо:

\vspace{4mm}

$
F_n^*(x) =
\begin{cases}
\begin{aligned}
&0,  \qquad & x \leqslant 0 \\
&0.41,  &0 < x \leqslant 1\\
&0.41 + 0.21 = 0.62, & 1 < x \leqslant 2\\
&0.41 + 0.21 + 0.2 = 0.82, & 2 < x \leqslant 3\\
&0.41 + 0.21 + 02 + 0.07 = 0.89, & 3 < x \leqslant 4\\
&0.41 + 0.21 + 02 + 0.07 + 0.05 = 0.94, & 4 < x \leqslant 5\\
&0.41 + 0.21 + 02 + 0.07 + 0.05  + 0.03 = 0.97, & 5 < x \leqslant 6\\
&0.41 + 0.21 + 02 + 0.07 + 0.05  + 0.03 + 0.01 = 0.98, & 6 < x \leqslant 7\\
&0.41 + 0.21 + 02 + 0.07 + 0.05  + 0.03 + 0.01  + 0.01 = 0.99, & 7 < x \leqslant 11\\
&1 , &x > 11\\
\end{aligned}
\end{cases}
$

\vspace{4mm}




\hypertarget{dfunc}{Побудуємо графік емпіричної функції розподілу}

\vspace{4mm}

\begin{tikzpicture}
	\begin{axis}[
		axis lines = middle,
		xmin = -1 , xmax = 13,
		ymin = -0.15, ymax = 1.05,
	%
	% тіки
		ytick = {0.41, 0.5, 1},
		xtick = {0, 1,2,3,4,5,6,7,11},
		extra x ticks = {0},
		extra x tick labels = {\!\!\!\!\!\!\! 0},
	%
	% підпис осей	
		xlabel style={below right},
		ylabel style={below left},
		xlabel = {$x$},
		ylabel = {$F_n^*(x)$},
		width = 17 cm,	
		height =14cm,
		title  = емпірична функція розподілу ,
		title style={yshift= - 13 cm}
	%
	]


\addplot[color = cyan, ultra thick, <-] coordinates {(0, 0.41 ) (1, 0.41)};
\addplot[color = cyan, ultra thick, <-] coordinates {(1, 0.62 ) (2, 0.62)};
\addplot[color = cyan, ultra thick, <-] coordinates {(2, 0.82 ) (3, 0.82)};
\addplot[color = cyan, ultra thick, <-] coordinates {(3, 0.89 ) (4, 0.89)};
\addplot[color = cyan, ultra thick, <-] coordinates {(4, 0.94 ) (5, 0.94)};
\addplot[color = cyan, ultra thick, <-] coordinates {(5, 0.97 ) (6, 0.97)};
\addplot[color = cyan, ultra thick, <-] coordinates {(6, 0.98 ) (7, 0.98)};
\addplot[color = cyan, ultra thick, <-] coordinates {(7, 0.99 ) (11, 0.99)};
\addplot[color = cyan, ultra thick, <-] coordinates {(11, 1 ) (13, 1)};
\addplot[color = cyan, ultra thick] coordinates {(-1, 0 ) (0, 0)};


\addplot[color = black, dashed] coordinates {(1, 0.62) (1, 0)};
\addplot[color = black, dashed] coordinates {(2, 0.82) (2, 0)};
\addplot[color = black, dashed] coordinates {(3, 0.89) (3, 0)};
\addplot[color = black, dashed] coordinates {(4, 0.94) (4, 0)};
\addplot[color = black, dashed] coordinates {(5, 0.97) (5, 0)};
\addplot[color = black, dashed] coordinates {(6, 0.98) (6, 0)};
\addplot[color = black, dashed] coordinates {(7, 0.99) (7, 0)};
\addplot[color = black, dashed] coordinates {(11, 1) (11, 0)};
\addplot[color = black, dashed] coordinates {(0,1) (11, 1)};

\end{axis}
\end{tikzpicture}

\newpage{}
\section{Знайти незміщену оцінку математичного сподівання та дисперсії.}

Як відомо з лекцій: незміщеною оцінкою математичного сподівання є вибіркове середнє \hyperlink{d1}{[1]}. А при невідомому математичному сподіванні незміщеною оцінкою дисперсії є виправлена вибіркова дисперсія \hyperlink{d1}{[1]}. Вибіркове середнє і виправлена вибіркова дисперсія задаються наступним \hypertarget{variance}{чином:}

$$ \bar \xi = \cfrac{1}{n} \sum_{i=1}^n \xi_i    \qquad \qquad \qquad \qquad  \mathbb{D}^{**} \xi = \cfrac{n}{n-1} \mathbb{D}^* \xi = \cfrac{1}{n-1} \sum_{i = 1}^{n} \left ( \xi_i - \bar \xi \right)^2, \quad (4.1)$$ де $ \mathbb{D}^* \xi$  \nolinebreak це вибіркова дисперсія.

Для нашої реалізації вибірки знайдемо значення вибіркового середнього та виправленої вибіркової дисперсії:

$$\bar x = \cfrac{1}{100} \sum_{i = 1}^{100} x_i =  \cfrac{1}{100} \left( 0\cdot41 + 1 \cdot 21 + 2 \cdot 20 + 3 \cdot 7 + 4 \cdot 5 + 5 \cdot 3 + 6 + 7 + 11  \right)= 1.41$$

$$ (\mathbb{D}^{**} \xi)_{\text{зн}} = \cfrac{1}{99} \sum_{i = 1}^{100} \left ( x_i - \bar x \right)^2 = 41 \cdot (0 - 1.41 )^2 + 21 \cdot (1 - 1.41)^2 + 20 \cdot (2 - 1.41)^2 + 7 \cdot (3-1.41)^2 + ... + (11 - 1.41)^2 =  3.29(48) $$


\section{Обчислити значення вибіркової медіани, моди, асиметрії.}

Знайдемо значення вибіркової медіани $\left( Me^* \xi \right)_{\text{зн}}$. У випадку побудови ДВР вона визначається як середня за розташуванням варіанта(або середнє арифметичне варіант, якщо їх парна кількість), тобто в нашому випадку маємо: $\left( Me^* \xi \right)_{\text{зн}} = x_5^* = 4$. Є ще один спосіб знайти значення вибіркової медіани, це знаходження першої варіанти, накопичена частість якої перевищила $0.5$. Тоді за \hyperlink{dfunc}{графіком ЕФР} маємо: $\left( Me^* \xi \right)_{\text{зн}} = x_1^* = 2$ 


\hspace{4mm}

Вибіркова мода у випадку побудови ДВР визначається як варінта з найбільшою частістю, тобто у нашому випадку: $\left( Mo^* \xi \right)_{\text{зн}} = x_1^* = 0$.

\hspace{4mm}

Вибіркова асиметрія визначається наступною формулою:

$$ As^* \xi = \cfrac{\frac{1}{n} \sum \limits_{i = 1}^n \left( \xi_i - \bar \xi \right)^3}{\left( \mathbb{D}^* \xi \right)^{3 /2}} $$

Для знаходження значення вибіркової асиметрії знайдемо значення вибіркової дисперсії. З \hyperlink{variance}{формули 4.1}, виправленої вибіркової дисперсії отримаємо:

$$\left( \mathbb{D}^* \xi \right)_{\text{зн}} = \cfrac{n-1}{n} \left( \mathbb{D}^{**} \xi \right)_{\text{зн}} = \cfrac{99}{100} \cdot 3.29(48) = 3.2619$$

Тепер можемо знайти значення вибркової асиметрії:

$$  \left( As^* \xi \right)_{\text{зн}} = \cfrac{\frac{1}{100} \sum \limits_{i = 1}^{100} \left( x_i - \bar x \right)^3}{\left( \mathbb{D}^* \xi \right)_{\text{зн}}^{3 /2}} = \cfrac{12.9489}{(3.2619)^{3/2}} = 2.198$$

\newpage{}

\section{Висунути гіпотезу про розподіл, за яким отримано вибірку.}

Розглянемо розподіл Паскаля. Нагадаємо, ДВВ $\zeta$ розподiлена за Законом Паскаля, якщо набуває значень $0, 1, 2, \dots$  з ймовiрностями:

$$ \mathbb{P}(\zeta = k) = \cfrac{1}{1+\theta} \left(\cfrac{\theta}{1+\theta} \right)^k, \quad \theta > 0$$
 
\begin{enumerate}

\item \hypertarget{pol_rosp}{Розглянемо полігон розподілу ймовірностей закону Паскаля для різних параметрів}  

\vspace{4mm}
\begin{tikzpicture}
	\begin{axis}[
		legend style={at={(1,0.9)},	
				    anchor=north east},
		axis lines = middle,
		xmin = -1 , xmax = 13,
		ymin = -0.15, ymax = 0.55,
	%
	% тіки
		ytick = {-0.2, -0.1, 0.1, 0.2, 0.3, 0.41, 0.5},
		xtick = {0, 1,2,3,4,5,6,7,11},
		extra x ticks = {0},
		extra x tick labels = {\!\!\!\!\!\!\! 0},
	%
	% підпис осей	
		xlabel style={below right},
		xlabel = {$x$},
		%ylabel = {$\omega$},
		width = 17 cm,	
		height =11cm,
		title  = {полігон відносних частот та полігон розподілу ймовірностей закону Паскаля  для $\theta=1$} ,
		title style={yshift= - 10.5 cm}
	%
	]

\addplot[color = cyan,  mark = *, ultra thick,  mark size = 2pt] coordinates {(0, 0.41) (1, 0.21) (2,0.2) (3, 0.07) (4,0.05) (5,0.03) (6, 0.01) (7,0.01) (11,0.01) };
\addplot[color = red,  mark = *, ultra thick,  mark size = 2pt] coordinates { (0,0.5) (1,0.25) (2,0.125) (3,0.0625) (4,0.03125) (5,0.015625) (6,0.0078125) (7,0.00390625) (8,0.001953125) (9,0.0009765625) (10,0.00048828125) (11,0.000244140625) };
		

\legend{полігон відносних частот, полігон розподілу ймовірностей закону Паскаля  для $\theta = 1$}

\addplot[color = black, dashed] coordinates {(1, 0) (1,0.25)};
\addplot[color = black, dashed] coordinates {(2, 0) (2,0.2) };
\addplot[color = black, dashed] coordinates {(3, 0) (3, 0.07)};
\addplot[color = black, dashed] coordinates {(4, 0) (4,0.05)};
\addplot[color = black, dashed] coordinates {(5, 0) (5,0.03)};

\node [rotate = 90 ] at (-0.9, 0.25) {частість / ймовірність};
\end{axis}
\end{tikzpicture}


\vspace{4mm}
\begin{tikzpicture}
	\begin{axis}[
		legend style={at={(1,0.9)},	
				    anchor=north east},
		axis lines = middle,
		xmin = -1 , xmax = 13,
		ymin = -0.15, ymax = 0.55,
	%
	% тіки
		ytick = {-0.2, -0.1, 0.1, 0.2, 0.3, 0.41, 0.5},
		xtick = {0, 1,2,3,4,5,6,7,11},
		extra x ticks = {0},
		extra x tick labels = {\!\!\!\!\!\!\! 0},
	%
	% підпис осей	
		xlabel style={below right},
		xlabel = {$x$},
		%ylabel = {$\omega$},
		width = 17 cm,	
		height =11cm,
		title  = {полігон відносних частот та полігон розподілу ймовірностей закону Паскаля  для $\theta=1.5$} ,
		title style={yshift= - 10.5 cm}
	%
	]

\addplot[color = cyan,  mark = *, ultra thick,  mark size = 2pt] coordinates {(0, 0.41) (1, 0.21) (2,0.2) (3, 0.07) (4,0.05) (5,0.03) (6, 0.01) (7,0.01) (11,0.01) };
\addplot[color = red,  mark = *, ultra thick,  mark size = 2pt] coordinates { (0,0.4) (1,0.24) (2,0.144) (3,0.0864) (4,0.05184) (5,0.031104) (6,0.0186624) (7,0.01119744) (8,0.006718464) (9,0.0040310784) (10,0.00241864704) (11,0.001451188224)};


\legend{полігон відносних частот, полігон розподілу ймовірностей закону Паскаля  для $\theta = 1.5$}

\addplot[color = black, dashed] coordinates {(1, 0) (1,0.24)};
\addplot[color = black, dashed] coordinates {(2, 0) (2,0.2) };
\addplot[color = black, dashed] coordinates {(3, 0) (3, 0.07)};
\addplot[color = black, dashed] coordinates {(4, 0) (4,0.05) };
\addplot[color = black, dashed] coordinates {(5, 0) (5,0.03)};

\node [rotate = 90 ] at (-0.9, 0.25) {частість / ймовірність};
\end{axis}
\end{tikzpicture}


\vspace{4mm}
\begin{tikzpicture}
	\begin{axis}[
		legend style={at={(1,0.9)},	
				    anchor=north east},
		axis lines = middle,
		xmin = -1 , xmax = 13,
		ymin = -0.15, ymax = 0.55,
	%
	% тіки
		ytick = {-0.2, -0.1, 0.1, 0.2, 0.3, 0.41, 0.5},
		xtick = {0, 1,2,3,4,5,6,7,11},
		extra x ticks = {0},
		extra x tick labels = {\!\!\!\!\!\!\! 0},
	%
	% підпис осей	
		xlabel style={below right},
		xlabel = {$x$},
		%ylabel = {$\omega$},
		width = 17 cm,	
		height =11cm,
		title  = {полігон відносних частот та полігон розподілу ймовірностей закону Паскаля  для $\theta=2$} ,
		title style={yshift= - 10.5 cm}
	%
	]

\addplot[color = cyan,  mark = *, ultra thick,  mark size = 2pt] coordinates {(0, 0.41) (1, 0.21) (2,0.2) (3, 0.07) (4,0.05) (5,0.03) (6, 0.01) (7,0.01) (11,0.01) };
\addplot[color = red,  mark = *, ultra thick,  mark size = 2pt] coordinates { (0,0.3333333333333333) (1,0.2222222222222222) (2,0.14814814814814814) (3,0.09876543209876543) (4,0.06584362139917696) (5,0.0438957475994513) (6,0.029263831732967534) (7,0.01950922115531169) (8,0.01300614743687446) (9,0.008670764957916306) (10,0.0057805099719442045) (11,0.0038536733146294698) };


\legend{полігон відносних частот, полігон розподілу ймовірностей закону Паскаля  для $\theta = 2$}

\addplot[color = black, dashed] coordinates {(1, 0) (1,0.2222222222222222)};
\addplot[color = black, dashed] coordinates {(2, 0) (2,0.2) };
\addplot[color = black, dashed] coordinates {(3, 0) (3,0.09876543209876543)};
\addplot[color = black, dashed] coordinates {(4, 0) (4,0.06584362139917696) };
\addplot[color = black, dashed] coordinates {(5, 0) (5,0.0438957475994513)};

\node [rotate = 90 ] at (-0.9, 0.25) {частість / ймовірність};

\end{axis}
\end{tikzpicture}

\textit{Зауваження} Червоні графіки не перетинають вісь абсцис. Це просто дефект відображення LaTeX. 

Отже, з графіків видно, що при певних значеннях параметра $\theta$(а саме 1, 1.5, 2) закону Паскаля його полігон розподілу схожий на полігон відносних частот нашої реалізації вибірки. 

\vspace{3mm}
\item \hypertarget{dfunc2}{Розглянемо графіки функцій розподілу Паскаля при різних значеннях параметра $\theta$}


\vspace{4mm}

\begin{tikzpicture}
	\begin{axis}[
		axis lines = middle,
		xmin = -1 , xmax = 13,
		ymin = -0.15, ymax = 1.3,
	%
	% тіки
		ytick = {0.41, 0.5, 1},
		xtick = {0, 1,2,3,4,5,6,7, 8, 9, 10, 11, 12},
		extra x ticks = {0},
		extra x tick labels = {\!\!\!\!\!\!\! 0},
	%
	% підпис осей	
		xlabel style={below right},
		ylabel style={below left},
		xlabel = {$x$},
		%ylabel = {$F_n^*(x)$},
		width = 17 cm,	
		height =13cm,
		title  = {ЕФР та функція розподілу Паскаля при $\theta = 1$} ,
		title style={yshift= - 12 cm}
	%
	]


\addplot[color = cyan, ultra thick, <-] coordinates {(0, 0.41 ) (1, 0.41)};
\addplot[color = cyan, ultra thick, <-] coordinates {(1, 0.62 ) (2, 0.62)};
\addplot[color = cyan, ultra thick, <-] coordinates {(2, 0.82 ) (3, 0.82)};
\addplot[color = cyan, ultra thick, <-] coordinates {(3, 0.89 ) (4, 0.89)};
\addplot[color = cyan, ultra thick, <-] coordinates {(4, 0.94 ) (5, 0.94)};
\addplot[color = cyan, ultra thick, <-] coordinates {(5, 0.97 ) (6, 0.97)};
\addplot[color = cyan, ultra thick, <-] coordinates {(6, 0.98 ) (7, 0.98)};
\addplot[color = cyan, ultra thick, <-] coordinates {(7, 0.99 ) (11, 0.99)};
\addplot[color = cyan, ultra thick, <-] coordinates {(11, 1 ) (13, 1)};
\addplot[color = cyan, ultra thick] coordinates {(-1, -0.000001 ) (0, -0.000001)};


\addplot[color = black, dashed] coordinates {(0,0.5) (0,0)};
\addplot[color = black, dashed] coordinates {(1,0.75) (1,0)};
\addplot[color = black, dashed] coordinates {(2,0.875) (2,0)};
\addplot[color = black, dashed] coordinates {(3,0.9375) (3,0)};
\addplot[color = black, dashed] coordinates {(4,0.96875) (4,0)};
\addplot[color = black, dashed] coordinates {(5,0.984375) (5,0)};
\addplot[color = black, dashed] coordinates {(6,0.9921875) (6,0)};
\addplot[color = black, dashed] coordinates {(7,0.99609375) (7,0)};
\addplot[color = black, dashed] coordinates {(8,0.998046875) (8,0)};
\addplot[color = black, dashed] coordinates {(9,0.9990234375) (9,0)};
\addplot[color = black, dashed] coordinates {(10,0.99951171875) (10,0)};
\addplot[color = black, dashed] coordinates {(11,0.999755859375) (11,0)};
\addplot[color = black, dashed] coordinates {(0,1) (11, 1)};


\addplot[color = red, ultra thick, <-] coordinates {(0,0.5) (1,0.5)};
\addplot[color = red, ultra thick, <-] coordinates {(1,0.75) (2,0.75)};
\addplot[color = red, ultra thick, <-] coordinates {(2,0.875) (3,0.875)};
\addplot[color = red, ultra thick, <-] coordinates {(3,0.9375) (4,0.9375)};
\addplot[color = red, ultra thick, <-] coordinates {(4,0.96875) (5,0.96875)};
\addplot[color = red, ultra thick, <-] coordinates {(5,0.984375) (6,0.984375)};
\addplot[color = red, ultra thick, <-] coordinates {(6,0.9921875) (7,0.9921875)};
\addplot[color = red, ultra thick, <-] coordinates {(7,0.99609375) (8,0.99609375)};
\addplot[color = red, ultra thick, <-] coordinates {(8,0.998046875) (9,0.998046875)};
\addplot[color = red, ultra thick, <-] coordinates {(9,0.9990234375) (10,0.9990234375)};
\addplot[color = red, ultra thick, <-] coordinates {(10,0.99951171875) (11,0.99951171875)};
\addplot[color = red, ultra thick, <-] coordinates {(11,0.999755859375) (12,0.999755859375)};
\addplot[color = red, thick] coordinates {(-1, 0.001 ) (0, 0.001)};


\node [rotate = 90 ] at (-0.86, 0.6) {Функція розподілу зак. Паскаля / ЕФР};

\legend{ЕФР,,,,,,,,,,,,,,,,,,,,,,,,,,,, Функція розподілу закону Паскаля при  $\theta = 1$}


\end{axis}
\end{tikzpicture}


\vspace{4mm}

\begin{tikzpicture}
	\begin{axis}[
		axis lines = middle,
		xmin = -1 , xmax = 13,
		ymin = -0.15, ymax = 1.3,
	%
	% тіки
		ytick = {0.41, 0.5, 1},
		xtick = {0, 1,2,3,4,5,6,7, 8, 9, 10, 11, 12},
		extra x ticks = {0},
		extra x tick labels = {\!\!\!\!\!\!\! 0},
	%
	% підпис осей	
		xlabel style={below right},
		ylabel style={below left},
		xlabel = {$x$},
		%ylabel = {$F_n^*(x)$},
		width = 17 cm,	
		height =13cm,
		title  = {ЕФР та функція розподілу Паскаля при $\theta = 1.5$} ,
		title style={yshift= - 12 cm}
	%
	]


\addplot[color = cyan, ultra thick, <-] coordinates {(0, 0.41 ) (1, 0.41)};
\addplot[color = cyan, ultra thick, <-] coordinates {(1, 0.62 ) (2, 0.62)};
\addplot[color = cyan, ultra thick, <-] coordinates {(2, 0.82 ) (3, 0.82)};
\addplot[color = cyan, ultra thick, <-] coordinates {(3, 0.89 ) (4, 0.89)};
\addplot[color = cyan, ultra thick, <-] coordinates {(4, 0.94 ) (5, 0.94)};
\addplot[color = cyan, ultra thick, <-] coordinates {(5, 0.97 ) (6, 0.97)};
\addplot[color = cyan, ultra thick, <-] coordinates {(6, 0.98 ) (7, 0.98)};
\addplot[color = cyan, ultra thick, <-] coordinates {(7, 0.99 ) (11, 0.99)};
\addplot[color = cyan, ultra thick, <-] coordinates {(11, 1 ) (13, 1)};
\addplot[color = cyan, ultra thick] coordinates {(-1, -0.000001 ) (0, -0.000001)};


\addplot[color = black, dashed] coordinates {(1, 0.64) (1, 0)};
\addplot[color = black, dashed] coordinates {(2, 0.82) (2, 0)};
\addplot[color = black, dashed] coordinates {(3, 0.89) (3, 0)};
\addplot[color = black, dashed] coordinates {(4, 0.94) (4, 0)};
\addplot[color = black, dashed] coordinates {(5, 0.97) (5, 0)};
\addplot[color = black, dashed] coordinates {(6, 0.98) (6, 0)};
\addplot[color = black, dashed] coordinates {(7, 0.99) (7, 0)};
\addplot[color = black, dashed] coordinates {(8,0.989922304) (8,0)};
\addplot[color = black, dashed] coordinates {(9,0.9939533824000001) (9,0)};
\addplot[color = black, dashed] coordinates {(10,0.9963720294400001) (10,0)};
\addplot[color = black, dashed] coordinates {(11, 1) (11, 0)};
\addplot[color = black, dashed] coordinates {(0,1) (11, 1)};

\addplot[color = red, ultra thick, <-] coordinates {(0,0.4) (1,0.4)};
\addplot[color = red, ultra thick, <-] coordinates {(1,0.64) (2,0.64)};
\addplot[color = red, ultra thick, <-] coordinates {(2,0.784) (3,0.784)};
\addplot[color = red, ultra thick, <-] coordinates {(3,0.8704000000000001) (4,0.8704000000000001)};
\addplot[color = red, ultra thick, <-] coordinates {(4,0.9222400000000001) (5,0.9222400000000001)};
\addplot[color = red, ultra thick, <-] coordinates {(5,0.9533440000000001) (6,0.9533440000000001)};
\addplot[color = red, ultra thick, <-] coordinates {(6,0.9720064) (7,0.9720064)};
\addplot[color = red, ultra thick, <-] coordinates {(7,0.98320384) (8,0.98320384)};
\addplot[color = red, ultra thick, <-] coordinates {(8,0.989922304) (9,0.989922304)};
\addplot[color = red, ultra thick, <-] coordinates {(9,0.9939533824000001) (10,0.9939533824000001)};
\addplot[color = red, ultra thick, <-] coordinates {(10,0.9963720294400001) (11,0.9963720294400001)};
\addplot[color = red, ultra thick, <-] coordinates {(11,0.9978232176640002) (12,0.9978232176640002)};
\addplot[color = red, thick] coordinates {(-1, 0.001 ) (0, 0.001)};


\node [rotate = 90 ] at (-0.86, 0.6) {Функція розподілу зак. Паскаля / ЕФР};

\legend{ЕФР,,,,,,,,,,,,,,,,,,,,,,,,,,,, Функція розподілу закону Паскаля при  $\theta = 1.5$}


\end{axis}
\end{tikzpicture}



\vspace{4mm}

\begin{tikzpicture}
	\begin{axis}[
		axis lines = middle,
		xmin = -1 , xmax = 13,
		ymin = -0.15, ymax = 1.3,
	%
	% тіки
		ytick = {0.41, 0.5, 1},
		xtick = {0, 1,2,3,4,5,6,7, 8, 9, 10, 11, 12},
		extra x ticks = {0},
		extra x tick labels = {\!\!\!\!\!\!\! 0},
	%
	% підпис осей	
		xlabel style={below right},
		ylabel style={below left},
		xlabel = {$x$},
		%ylabel = {$F_n^*(x)$},
		width = 17 cm,	
		height =13cm,
		title  = {ЕФР та функція розподілу Паскаля при $\theta = 1.5$} ,
		title style={yshift= - 12 cm}
	%
	]


\addplot[color = cyan, ultra thick, <-] coordinates {(0, 0.41 ) (1, 0.41)};
\addplot[color = cyan, ultra thick, <-] coordinates {(1, 0.62 ) (2, 0.62)};
\addplot[color = cyan, ultra thick, <-] coordinates {(2, 0.82 ) (3, 0.82)};
\addplot[color = cyan, ultra thick, <-] coordinates {(3, 0.89 ) (4, 0.89)};
\addplot[color = cyan, ultra thick, <-] coordinates {(4, 0.94 ) (5, 0.94)};
\addplot[color = cyan, ultra thick, <-] coordinates {(5, 0.97 ) (6, 0.97)};
\addplot[color = cyan, ultra thick, <-] coordinates {(6, 0.98 ) (7, 0.98)};
\addplot[color = cyan, ultra thick, <-] coordinates {(7, 0.99 ) (11, 0.99)};
\addplot[color = cyan, ultra thick, <-] coordinates {(11, 1 ) (13, 1)};
\addplot[color = cyan, ultra thick] coordinates {(-1, -0.000001 ) (0, -0.000001)};


\addplot[color = black, dashed] coordinates {(1,0.62 ) (1, 0)};
\addplot[color = black, dashed] coordinates {(2, 0.82) (2, 0)};
\addplot[color = black, dashed] coordinates {(3, 0.89) (3, 0)};
\addplot[color = black, dashed] coordinates {(4, 0.94) (4, 0)};
\addplot[color = black, dashed] coordinates {(5, 0.97) (5, 0)};
\addplot[color = black, dashed] coordinates {(6, 0.98) (6, 0)};
\addplot[color = black, dashed] coordinates {(7, 0.99) (7, 0)};
\addplot[color = black, dashed] coordinates {(8,0.989922304) (8,0)};
\addplot[color = black, dashed] coordinates {(9,0.9939533824000001) (9,0)};
\addplot[color = black, dashed] coordinates {(10,0.9963720294400001) (10,0)};
\addplot[color = black, dashed] coordinates {(11, 1) (11, 0)};
\addplot[color = black, dashed] coordinates {(0,1) (11, 1)};

\addplot[color = red, ultra thick, <-] coordinates {(0,0.3333333333333333) (1,0.3333333333333333)};
\addplot[color = red, ultra thick, <-] coordinates {(1,0.5555555555555556) (2,0.5555555555555556)};
\addplot[color = red, ultra thick, <-] coordinates {(2,0.7037037037037037) (3,0.7037037037037037)};
\addplot[color = red, ultra thick, <-] coordinates {(3,0.8024691358024691) (4,0.8024691358024691)};
\addplot[color = red, ultra thick, <-] coordinates {(4,0.8683127572016461) (5,0.8683127572016461)};
\addplot[color = red, ultra thick, <-] coordinates {(5,0.9122085048010974) (6,0.9122085048010974)};
\addplot[color = red, ultra thick, <-] coordinates {(6,0.9414723365340649) (7,0.9414723365340649)};
\addplot[color = red, ultra thick, <-] coordinates {(7,0.9609815576893767) (8,0.9609815576893767)};
\addplot[color = red, ultra thick, <-] coordinates {(8,0.9739877051262511) (9,0.9739877051262511)};
\addplot[color = red, ultra thick, <-] coordinates {(9,0.9826584700841674) (10,0.9826584700841674)};
\addplot[color = red, ultra thick, <-] coordinates {(10,0.9884389800561116) (11,0.9884389800561116)};
\addplot[color = red, ultra thick, <-] coordinates {(11,0.9922926533707411) (12,0.9922926533707411)};
\addplot[color = red, thick] coordinates {(-1, 0.001 ) (0, 0.001)};


\node [rotate = 90 ] at (-0.86, 0.6) {Функція розподілу зак. Паскаля / ЕФР};

\legend{ЕФР,,,,,,,,,,,,,,,,,,,,,,,,,,,, Функція розподілу закону Паскаля при  $\theta = 1.5$}


\end{axis}
\end{tikzpicture}

\textit{Зауваження} Червоні графіки не дотикаються до лінії $y = 1$. Це також дефект відображення LaTeX

\newpage{}

Як бачимо, графік емпіричної функції розподілу закона Паскаля при певних параметрах $\theta$(а саме при 1, 1.5, 2) схожий на графік емпіричної функції розподілу реалізації вибірки.

\item При будь-яких значеннях параметра $\theta$ мода випадкової величини розподіленої за законом Паскаля дорівнює нулю, в той же час значення вибіркової моди для нашої реалізації дорівнює нулю. 

\item Відомо що, якщо $\zeta \sim Pas\left( \theta \right)$, то $\mathbb{E} \zeta = \theta, \mathbb{D} \zeta = \theta + \theta^2$. Якщо замість параметру $\theta$ взяти значення вибіркового середнього(далі буде показано, що вибіркове середнє це незміщена, конзистентна та ефективна точкова оцінка невідомого параметру $\theta$) , то величини $\left( \mathbb{D}^{**} \xi \right)_{\text{зн}}$ і $\bar x + (\bar x)^2$  будуть досить схожими(3.29(48) і 3.3981 відповідно)

\item \hypertarget{link1}{Також, якщо взяти заміть парметру $\theta$ значення вибіркового середнього, то полігон відносних частот і полігон розподілу ймовірностей закону Паскаля будуть досить схожими. Також при такому значенні параметра будуть дуже схожі графіки ЕФР і функція розподілу закону Паскаля.(див. наступні графіки)}



\vspace{4mm}
\begin{tikzpicture}
	\begin{axis}[
		legend style={at={(1,0.9)},	
				    anchor=north east},
		axis lines = middle,
		xmin = -1 , xmax = 13,
		ymin = -0.15, ymax = 0.55,
	%
	% тіки
		ytick = {-0.2, -0.1, 0.1, 0.2, 0.3, 0.41, 0.5},
		xtick = {0, 1,2,3,4,5,6,7,11},
		extra x ticks = {0},
		extra x tick labels = {\!\!\!\!\!\!\! 0},
	%
	% підпис осей	
		xlabel style={below right},
		xlabel = {$x$},
		%ylabel = {$\omega$},
		width = 17 cm,	
		height =11cm,
		title  = {полігон відносних частот та полігон розподілу ймовірностей закону Паскаля  для $\theta=1.41$} ,
		title style={yshift= - 10.5 cm}
	%
	]

\addplot[color = cyan,  mark = *, ultra thick,  mark size = 2pt] coordinates {(0, 0.41) (1, 0.21) (2,0.2) (3, 0.07) (4,0.05) (5,0.03) (6, 0.01) (7,0.01) (11,0.01) };
\addplot[color = red,  mark = *, ultra thick,  mark size = 2pt] coordinates { (0,0.41493775933609955) (1,0.24276441521323666) (2,0.14203229271811768) (3,0.08309773142429291) (4,0.04861734494118381) (5,0.02844417276641874) (6,0.01664161145255204) (7,0.009736378484688123) (8,0.005696387412203424) (9,0.003332741183073372) (10,0.0019498610241217653) (11,0.0011407900597558876)};


\legend{полігон відносних частот, полігон розподілу ймовірностей закону Паскаля  для $\theta = 1.41$}

\addplot[color = black, dashed] coordinates {(1, 0) (1,0.2222222222222222)};
\addplot[color = black, dashed] coordinates {(2, 0) (2,0.2) };
\addplot[color = black, dashed] coordinates {(3, 0) (3,0.09876543209876543)};
\addplot[color = black, dashed] coordinates {(4, 0) (4,0.06584362139917696) };
\addplot[color = black, dashed] coordinates {(5, 0) (5,0.0438957475994513)};

\node [rotate = 90 ] at (-0.9, 0.25) {частість / ймовірність};

\end{axis}
\end{tikzpicture}


\vspace{4mm}

\begin{tikzpicture}
	\begin{axis}[
		axis lines = middle,
		xmin = -1 , xmax = 13,
		ymin = -0.15, ymax = 1.3,
	%
	% тіки
		ytick = {0.41, 0.5, 1},
		xtick = {0, 1,2,3,4,5,6,7, 8, 9, 10, 11, 12},
		extra x ticks = {0},
		extra x tick labels = {\!\!\!\!\!\!\! 0},
	%
	% підпис осей	
		xlabel style={below right},
		ylabel style={below left},
		xlabel = {$x$},
		%ylabel = {$F_n^*(x)$},
		width = 17 cm,	
		height =13cm,
		title  = {ЕФР та функція розподілу Паскаля при $\theta = 1.41$} ,
		title style={yshift= - 12 cm}
	%
	]


\addplot[color = cyan, ultra thick, <-] coordinates {(0, 0.41 ) (1, 0.41)};
\addplot[color = cyan, ultra thick, <-] coordinates {(1, 0.62 ) (2, 0.62)};
\addplot[color = cyan, ultra thick, <-] coordinates {(2, 0.82 ) (3, 0.82)};
\addplot[color = cyan, ultra thick, <-] coordinates {(3, 0.89 ) (4, 0.89)};
\addplot[color = cyan, ultra thick, <-] coordinates {(4, 0.94 ) (5, 0.94)};
\addplot[color = cyan, ultra thick, <-] coordinates {(5, 0.97 ) (6, 0.97)};
\addplot[color = cyan, ultra thick, <-] coordinates {(6, 0.98 ) (7, 0.98)};
\addplot[color = cyan, ultra thick, <-] coordinates {(7, 0.99 ) (11, 0.99)};
\addplot[color = cyan, ultra thick, <-] coordinates {(11, 1 ) (13, 1)};
\addplot[color = cyan, ultra thick] coordinates {(-1, -0.000001 ) (0, -0.000001)};


\addplot[color = black, dashed] coordinates {(1,0.62 ) (1, 0)};
\addplot[color = black, dashed] coordinates {(2, 0.82) (2, 0)};
\addplot[color = black, dashed] coordinates {(3, 0.89) (3, 0)};
\addplot[color = black, dashed] coordinates {(4, 0.94) (4, 0)};
\addplot[color = black, dashed] coordinates {(5, 0.97) (5, 0)};
\addplot[color = black, dashed] coordinates {(6, 0.98) (6, 0)};
\addplot[color = black, dashed] coordinates {(7, 0.99) (7, 0)};
\addplot[color = black, dashed] coordinates {(8,0.989922304) (8,0)};
\addplot[color = black, dashed] coordinates {(9,0.9939533824000001) (9,0)};
\addplot[color = black, dashed] coordinates {(10,0.9963720294400001) (10,0)};
\addplot[color = black, dashed] coordinates {(11, 1) (11, 0)};
\addplot[color = black, dashed] coordinates {(0,1) (11, 1)};

\addplot[color = red, ultra thick, <-] coordinates {(0,0.41493775933609955) (1,0.41493775933609955)};
\addplot[color = red, ultra thick, <-] coordinates {(1,0.6577021745493362) (2,0.6577021745493362)};
\addplot[color = red, ultra thick, <-] coordinates {(2,0.7997344672674539) (3,0.7997344672674539)};
\addplot[color = red, ultra thick, <-] coordinates {(3,0.8828321986917468) (4,0.8828321986917468)};
\addplot[color = red, ultra thick, <-] coordinates {(4,0.9314495436329306) (5,0.9314495436329306)};
\addplot[color = red, ultra thick, <-] coordinates {(5,0.9598937163993494) (6,0.9598937163993494)};
\addplot[color = red, ultra thick, <-] coordinates {(6,0.9765353278519014) (7,0.9765353278519014)};
\addplot[color = red, ultra thick, <-] coordinates {(7,0.9862717063365896) (8,0.9862717063365896)};
\addplot[color = red, ultra thick, <-] coordinates {(8,0.991968093748793) (9,0.991968093748793)};
\addplot[color = red, ultra thick, <-] coordinates {(9,0.9953008349318664) (10,0.9953008349318664)};
\addplot[color = red, ultra thick, <-] coordinates {(10,0.9972506959559881) (11,0.9972506959559881)};
\addplot[color = red, ultra thick, <-] coordinates {(11,0.998391486015744) (12,0.998391486015744)};
\addplot[color = red, thick] coordinates {(-1, 0.001 ) (0, 0.001)};


\node [rotate = 90 ] at (-0.86, 0.6) {Функція розподілу зак. Паскаля / ЕФР};

\legend{ЕФР,,,,,,,,,,,,,,,,,,,,,,,,,,,, Функція розподілу закону Паскаля при  $\theta = 1.41$}


\end{axis}
\end{tikzpicture}

\end{enumerate}

\hypertarget{s}{Отже, підсумуємо, що отримали}

\begin{itemize}

\item Полігон відносних частот реалізації вибірки схожий на полігон розподілу ймовірностей закону Паскаля.\hyperlink{pol_rosp}{(див. ст. 5-6 )}

\item Графік емпіричної функції розподілу нашої реалізації вибірки схожий на графік фунції розподілу закону Паскаля \hyperlink{dfunc2}{(див. ст. 6-7)}


\item При будь-яких значеннях параметра $\theta$ мода випадкової величини розподіленої за законом Паскаля дорівнює нулю, в той же час значення вибіркової моди для нашої реалізації дорівнює нулю. 


\item Відомо що, якщо $\zeta \sim Pas\left( \theta \right)$, то $\mathbb{E} \zeta = \theta, \mathbb{D} \zeta = \theta + \theta^2$. Якщо замість параметру $\theta$ взяти значення вибіркового середнього(далі буде показано, що вибіркове середнє це незміщена, конзистентна та ефективна точкова оцінка невідомого параметру $\theta$) , то величини $\left( \mathbb{D}^{**} \xi \right)_{\text{зн}}$ і $\bar x + (\bar x)^2$  будуть досить схожими(3.29(48) і 3.3981 відповідно)

\item Також, якщо взяти заміть парметру $\theta$ значення вибіркового середнього, то полігон відносних частот і полігон розподілу ймовірностей закону Паскаля будуть досить схожими. Також при такому значенні параметра будуть дуже схожі графіки ЕФР і функція розподілу закону Паскаля. \hyperlink{link1}{(див. ст. 8)}

\end{itemize}
 


\noindent\rule{4cm}{0.4pt}

Таким чином, на основі вище написаних тверджень мною висувається гіпотеза, що генеральна сукупність, якою породжена данна вибірка, розподілена за законом Паскаля.


\newpage{}

\section{Знайти точкову оцінку параметрів гіпотетичного закону розподілу та перевірити їх властивості.}

Знайдемо точкові оцінки невідомого параметра $\theta$ двома способами:

\begin{enumerate}

\item Метод моментів:

Відомо, що для $\xi \sim Pas(\theta):  \mathbb{E} \xi = \theta$. Замінивши початковий момент першого порядку на емпіричний початковий момент першого порядку отримаємо рівняння Пірсона:

$$ (\theta^*)_{\text{мм}} = \mathbb{E^*} \xi \quad \Rightarrow{} \quad (\theta^*)_{\text{мм}} = \cfrac{1}{n} \sum \limits_{i=1}^{n} \xi_i = \bar \xi$$



\item Метод максимальної правдоподібності:

Спочатку знайдемо функцію правдоподібності закону Паскаля:

$$ \mathcal{L}(\vec x; \theta) = \prod \limits_{k = 1}^{n} \mathbb{P}\{ \xi = x_k \} = \prod \limits_{k =1 }^{n} \cfrac{1}{1+\theta} \left(
\cfrac{\theta}{1 + \theta} \right)^{x_k} = \left(\cfrac{1}{1+\theta} \right)^n \left(\cfrac{\theta}{1+\theta} \right)^{\sum \limits_{k=1}^{n} x_k} = 
\cfrac{\theta^{\sum \limits_{k=1}^{n} x_k}}{\left( 1 + \theta\right)^{n + \sum \limits_{k=1}^{n} x_k}}$$

Тепер знайдемо логарифмічну функцію правдоподібності закону Паскаля:

$$ \ln \mathcal{L} (\vec x; \theta) = \ln{\cfrac{\theta^{\sum \limits_{k=1}^{n} x_k}}{\left( 1 + \theta\right)^{n + \sum \limits_{k=1}^{n} x_k}}} 
= \ln \theta \cdot \sum \limits_{k=1}^{n} x_k - \ln (1+\theta) \cdot \left(n + \sum \limits_{k=1}^{n} x_k\right)$$
 
\hypertarget{derr_lfunc}{Тепер знайдемо частинну похідну} $\cfrac{\partial \ln \mathcal{L} (\vec x; \theta)}{\partial \theta}$ і прирівняємо її до нуля(таким чином отримаємо рівняння правдоподібності):

$$ \cfrac{\partial \ln \mathcal{L} (\vec x; \theta)}{\partial \theta} = \cfrac{1}{\theta} \sum \limits_{k=1}^{n} x_k - \cfrac{\left(n + \sum \limits_{k=1}^{n} x_k\right)}{1+\theta} = \cfrac{\sum \limits_{k=1}^{n} x_k  + \theta \cdot \sum \limits_{k=1}^{n} x_k  - \theta \cdot n - \theta \cdot \sum \limits_{k=1}^{n} x_k }{\theta (1+\theta)} = \cfrac{- \theta \cdot n + \sum \limits_{k=1}^{n} x_k }{\theta (1 + \theta)} = 0$$

Тоді:

$$- \theta \cdot n + \sum \limits_{k=1}^{n} x_k  = \quad \Rightarrow{} \quad \theta = \cfrac{1}{n} \sum \limits_{k=1}^{n} x_k$$


Отримали $\theta_{\text{кр}} =  \frac{1}{n} \sum \limits_{k=1}^{n} x_k $ - критична точка функції $ \ln \mathcal{L} (\vec x; \theta)$. Тепер перевіримо чи є ця точка саме максимумом цієї функції. Для цього має виконуватись умова:

$$\left. \cfrac{\partial^2  \ln \mathcal{L} (\vec x; \theta)}{\partial \theta^2} \right|_{\theta =\frac{1}{n} \sum \limits_{k=1}^{n} x_k} < 0$$


\newpage{}

маємо:

$$ \cfrac{\partial^2  \ln \mathcal{L} (\vec x; \theta)}{\partial \theta^2} = \cfrac{-n \cdot (\theta + \theta^2) - (1+2 \theta) \cdot \left( - \theta \cdot n + \sum \limits_{k=1}^{n} x_k  \right)}{\theta^2 (1+ \theta)^2} = \cfrac{- n \cdot \theta^2  -  \sum \limits_{k=1}^{n} x_k + 2 n \cdot \theta^2  - 2\theta  \sum \limits_{k=1}^{n} x_k  }{\theta^2 (1+ \theta)^2}$$


$$\left. \cfrac{\partial^2  \ln \mathcal{L} (\vec x; \theta)}{\partial \theta^2} \right|_{\theta =\frac{1}{n} \sum \limits_{k=1}^{n} x_k}  = 
\frac{-\frac{1}{n} \left(\sum \limits_{k=1}^{n} x_k \right)^2 - \sum \limits_{k=1}^{n} x_k  + \frac{2}{n} \left(\sum \limits_{k=1}^{n} x_k \right)^2 - \frac{2}{n} \left(\sum \limits_{k=1}^{n} x_k \right)^2}{\left(\frac{1}{n}\sum \limits_{k=1}^{n} x_k \right)^2 \cdot \left(1 + \frac{1}{n}\sum \limits_{k=1}^{n} x_k \right)^2} = \cfrac{-\left(\frac{1}{n} \left(\sum \limits_{k=1}^{n} x_k \right)^2 + \sum \limits_{k=1}^{n} x_k \right)}{\left(\frac{1}{n}\sum \limits_{k=1}^{n} x_k \right)^2 \cdot \left(1 + \frac{1}{n}\sum \limits_{k=1}^{n} x_k \right)^2}$$

\noindent\rule{4cm}{0.4pt}

\hspace{5mm}

Оскільки $\sum \limits_{k=1}^{n} x_k > 0, \quad n > 0$, то :

$$ \left. \cfrac{\partial^2  \ln \mathcal{L} (\vec x; \theta)}{\partial \theta^2} \right|_{\theta =\frac{1}{n} \sum \limits_{k=1}^{n} x_k} < 0 $$

Отримали, що $\theta_{\text{кр}} =  \frac{1}{n} \sum \limits_{k=1}^{n} x_k $ - точка максимуму  функції $ \ln \mathcal{L} (\vec x; \theta)$. Отже, отримали точкову оцінку невідомого параметру $\theta$  розподілу Паскаля за методом максимальної правдоподібності:

$$ \left( \theta^* \right)_{\text{ммп}} = \sum \limits_{k=1}^{n} \xi_k = \bar \xi$$

\end{enumerate}

Обома методами отримали однакові точкові оцінки невідомого параметру $\theta:$

$$\theta^* =  (\theta^*)_{\text{мм}} = \left( \theta^* \right)_{\text{ммп}} = \cfrac{1}{n} \sum \limits_{k=1}^{n} \xi_k = \bar \xi $$

\noindent\rule{4cm}{0.4pt}

\hspace{5mm}

Перевіримо властивості точкової оцінки $\theta^*:$ 


\begin{enumerate}

\item \textbf{Незміщеність}

$$ \mathbb{E} \theta^* = \mathbb{E} \cfrac{1}{n} \sum \limits_{k=1}^{n} \xi_k  = \cfrac{1}{n} \sum \limits_{k=1}^{n} \mathbb{E} \xi_k  = \Biggl| \text{всі $\xi_k$ розподілені як $\xi$} \Biggl| = \cfrac{1}{n} \cdot n \cdot \mathbb{E} \xi = \theta$$

Отже, точкова оцінка $\theta^*$ - незміщена


\item \textbf{Конзистентність}

$\{ \xi_k \}^{n}_{k=1}$ - незалежні та однаково розподілені випадкові величини. Для них існують математичні сподівання $\mathbb{E} \xi_k = \theta$ та дисперсії $\mathbb{D} \xi_k = \theta + \theta^2$. При чому всі дисперсії - рівномірно обмежені. Тоді за Законом Великих Чисел у формі Чебишова маємо:

$$ \cfrac{1}{n} \sum_{k=1}^{n} \xi_k \xrightarrow{\text{P}} \cfrac{1}{n} \sum \limits_{k=1}^{n} \mathbb{E} \xi_k, \quad n \to \infty \quad \Rightarrow{} \quad \Biggl|  \cfrac{1}{n} \sum_{k=1}^{n} = \theta^*;  \cfrac{1}{n} \sum \limits_{k=1}^{n} \mathbb{E} \xi_k = \theta \Biggl|  \quad \Rightarrow{} \quad \theta^* \xrightarrow{\text{P}} \theta, \quad n \to \infty $$


Отже, точкова оцінка - конзистентна. До речі, за Посиленим законом Великих Чисел точкова оцінка $\theta^*$ є навіть сильно конзистентною, тобто $ \theta^* \xrightarrow{\text{P1}} \theta,  n \to \infty $.

\item \textbf{Ефективність}

На \hyperlink{derr_lfunc}{ст. 10} Отримали значення виразу: 

$$ \cfrac{\partial \ln \mathcal{L} (\vec x; \theta)}{\partial \theta} =  \cfrac{- \theta \cdot n + \sum \limits_{k=1}^{n} x_k }{\theta (1 + \theta)} $$

Тепер перейдемо до випадкової вибірки:

$$ \cfrac{\partial \ln \mathcal{L} (\vec \xi; \theta)}{\partial \theta} =  \cfrac{- \theta \cdot n + \sum \limits_{k=1}^{n} \xi_k }{\theta (1 + \theta)} 
= \cfrac{1}{\theta(1+\theta) } \cdot \cfrac{1}{n} \left( \theta^* - \theta \right) = C(n, \theta) \cdot(\theta^*-\theta)$$

Отже, за наслідком з нерівності Рао-Крамера оцінка $\theta^* = \bar \xi$ є ефективною.


\item \hypertarget{asd}{\textbf{Асимптотична нормальність}}



$$ \cfrac{\theta^* - \theta}{\sqrt{\mathbb{D} \theta^*}} = \cfrac{\frac{1}{n} \sum \limits_{k = 1}^{n} \xi_k - \frac{1}{n} \cdot n \cdot \mathbb{E} \xi}{\sqrt{\mathbb{D}\frac{1}{n} \sum \limits_{k=1}^{n} \xi_k }} = \Biggl| \text{всі $\xi_k$ незалежні та однаково розподілені} \Biggl| = 
\cfrac{\frac{1}{n} \sum \limits_{k=1}^{n} \left( \xi_k - \mathbb{E} \xi_k \right)}{\sqrt{\frac{1}{n^2} \cdot \sum \limits_{k=1}^{n} \mathbb{D} \xi_k}} = $$

$$ = \cfrac{\sum \limits_{k=1}^{n} \left( \xi_k - \mathbb{E} \xi_k \right)}{\sum \limits_{k=1}^{n}\sigma_k^2} \xrightarrow{F} \nu \sim N(0,1), n \to \infty$$

Останній граничний перехід був записаний за наслідком до теореми Ляпунова: "Якщо всі $\xi_k$ однаково розподілені, то умова Ляпунова виконується автоматично".

\end{enumerate}

\noindent\rule{4cm}{0.4pt}

\hspace{5mm}

Підсумуємо. В цьому розділі ми знайшли точкову оцінку невідомого параметру закону Паскаля: $\theta^* = \bar \xi$, а  також довели, шо ця точкова оцінка є незміщеною, конзистентною, ефективною, а також асимптотично нормальною. 



\section{Перевірити за допомогою критерію $\chi^2$ (Пірсона) гіпотезу про розподіл з рівнем значущості  $\alpha = 0.05$.}

Критерій $\chi^2$ опирається на теорему Пірсона. Вона формулюється так:

\begin{theorem}
(теорема Пірсона) 

\begin{enumerate}
\item Якщо \textbf{проста} гіпотеза $H_0$ щодо закону розподілу ГС справджується, то статистика $\eta$ прямує за розподілом до розподілу $\chi_{r-1}^2$ (розподіл $\chi^2$ з $r-1$ ступенями вільності) при $n \to \infty$

\item Якщо \textbf{складна} гіпотеза $H_0$ щодо закону розподілу ГС справджується, то статистика $\eta$ прямує за розподілом до розподілу $\chi_{r-s-1}^2$(розподіл $\chi^2$ з $r-s-1$ ступенями вільності) при $n \to \infty$. Тут $s$ - кількість невідомих параметрів розподілу, які оцінюються.
\end{enumerate} 

Статистика $\eta$ визначається як:

$$ \eta = \sum \limits_{i=1}^{r} \cfrac{n}{p_i} \left( \cfrac{n_i}{n} - p_i \right)^2 = \sum \limits_{i=1}^{r} \cfrac{\left( n_i - np_i \right)^2}{np_i} $$ 

\end{theorem}

Згідно з \hyperlink{s}{висновками} розділу 6 висувається гіпотеза $H_0: \xi \sim Pas(1.41)$. За нашою гіпотезою генеральна сукупність може приймати такі значення $X = \{0, 1, 2, 3 \dots \}$. Розіб'ємо множину $X$ на $r$ підмножин за таким правилом:

\begin{enumerate}

\item Якщо $r \geqslant 20$, то потрібно, щоб виконувалсь $np_i \geqslant 5$

\item Якщо $r < 20$, то потрібно, щоб виконувалсь $np_i \geqslant 20$

\end{enumerate}

Отже, маємо таке розбиття $X$:
\vspace{5mm}
$$X_0 = \{0\}; \quad X_1 = \{1\}; \quad X_2 = \{2 \};  \quad X_3 = \{3,4,5 \dots \}$$

\vspace{5mm}
Обчислимо ймовірності $p_i = \mathbb{P} \{\xi \in X / H_0\}$. тут $n_i$ це кількість значень з реалізації вибірки, які попали в множину $X_i$. 

\begin{center}
\begin{tabular}{| c | c | c | c | c |}
\hline
$X_i$ & \{0\} & \{1\} & \{2\} & \{3, 4 , 5 ...\} \\ \hline
$p_i$ & 0.41494 & 0.24276 & 0.14203 & 0.20027 \\ \hline
$np_i$ & 41.494 & 24.276 & 14.203 & 20.027 \\ \hline
$n_i$ & 41 & 21 & 20 & 18 \\ \hline
\end{tabular}
\end{center}

Виконаємо перевірку для значень $p_i$:

$$ \sum \limits_{i=1}^{4} p_i = 0.41494 + 0.24276 + 0.14203 + 0.20027 = 1$$


Тепер обчислимо значення статистики $\eta$ взятої з теореми Пірсона:

$$ \eta_{\text{зн}} = \cfrac{(41 - 41.494)^2}{41.494} + \cfrac{(21-24.276)^2}{24.276} + \cfrac{(20 - 14.203)^2}{14.203} + \cfrac{(18 - 20.027)^2}{20.027} \approx 3.0191$$

\vspace{4mm}

Тепер за таблицею розподілу Пірсона знайдемо значення $t_{\text{кр}}$. $ \quad r-s-1 = 4 - 1 - 1 = 2$, тому оскільки $\alpha = 0.05$ маємо:
$t_{\text{кр}} = 5.99$. Отримали $\eta_{\text{кр}} < t_{\text{кр}}$, отже, на рівні значущості 0.05 дані не суперечать висунутій гіпотезі про те, що генеральна сукупність розподілена за законом Паскаля$(\xi \sim Pas(1.41))$.

\section{Знайти довірчий інтервал для параметрів гіпотетичного закону розподілу, взяв рівень надійності $\gamma = 0.95$.}

Побудуємо довірчий інтерал такий, що: $\mathbb{P} \left\{ |\theta^* - \theta| < \varepsilon \right\} = \gamma = 0.95$. В розділі 6 було \hyperlink{asd}{доведено}, що точкова оцінка $\theta^* = \bar \xi$ є асимптотично нормальною, тому розподіл $\frac{\theta^*-\theta}{\sqrt{\mathbb{D} \theta}}$ можна вважати приблизно $N(0,1)$. Тоді довірчий інтервал можна знайти з наступної рівності:

$$ \mathbb{P} \left\{ \cfrac{|\theta^*-\theta|}{\sqrt{\mathbb{D} \theta^*}} < t_{\gamma}\right\} \approx 2 \Phi \left( t_{\gamma}\right)$$

Знайдемо $\mathbb{D} \theta^*:$

$$ \mathbb{D} \theta^* = \mathbb{D} \cfrac{1}{n} \sum \limits_{k=1}^{n} \xi_k = \Biggl| \text{всі $\xi_k$ - незалежні та однаково розподілені} \Biggl| = \cfrac{1}{n^2} \cdot n \cdot \mathbb{D} \xi = \cfrac{1}{n}\cdot \left(\theta + \theta^2 \right)$$

\newpage{}

За допомогою таблиці значень функції Лапласа знайдемо таке $t_{\gamma}$, щоб $2\Phi(t_{\gamma}) \approx 0.95:  \quad t_{\gamma} \approx 1.44$

Тепер розв'яжемо нерівність $ \cfrac{|\theta^*-\theta|}{\sqrt{\mathbb{D} \theta^*}} < t_{\gamma}$ відносно $\theta$: 


$$ \cfrac{|\theta^*-\theta|}{\sqrt{\mathbb{D} \theta^*}} < t_{\gamma} \quad \Rightarrow{} \quad \cfrac{\left( \theta^* - \theta \right)^2}{\mathbb{D} \theta^*} < t_{\gamma}^2 \quad \Rightarrow{} \quad \left( \theta^* - \theta \right)^2 < t_{\gamma}^2 \cdot \cfrac{1}{n}\cdot \left(\theta + \theta^2 \right)$$

$$ (\theta^*)^2 - 2 \cdot \theta \cdot \theta^* + \theta^2 < \cfrac{t_{\gamma}^2}{n} \cdot \theta + \cfrac{t_{\gamma}^2}{n} \cdot \theta^2$$

$$ \left( 1 - \cfrac{t_{\gamma}^2}{n} \right)\cdot \theta^2 - \left( 2 \cdot \theta^* + \cfrac{t_{\gamma}^2}{n} \right) \cdot \theta+ \left( \theta^* \right)^2  < 0 $$

$$ D = \left( 2 \cdot \theta^* + \cfrac{t_{\gamma}^2}{n} \right)^2 - 4 \cdot  \left( 1 - \cfrac{t_{\gamma}^2}{n} \right) \cdot  \left( \theta^* \right)^2 =
4 \cdot \theta^* \cdot \cfrac{t_{\gamma}^2}{n} +\cfrac{t_{\gamma}^4}{n^2} + \cfrac{t_{\gamma}^2}{n} \cdot \left( \theta^* \right)^2$$

\vspace{4mm}

Отримали такий довірчий інтервал з рівнем надійності $\gamma$:

$$ \left( \cfrac{2 \cdot \theta^* + \frac{t_{\gamma}^2}{n} - \sqrt{4 \cdot \theta^* \cdot \cfrac{t_{\gamma}^2}{n} +\cfrac{t_{\gamma}^4}{n^2} + \cfrac{t_{\gamma}^2}{n} \cdot \left( \theta^* \right)^2}}{2 \cdot \left(1 - \frac{t_{\gamma}^2}{n} \right)} ; \quad \cfrac{2 \cdot \theta^* + \frac{t_{\gamma}^2}{n} + \sqrt{4 \cdot \theta^* \cdot \cfrac{t_{\gamma}^2}{n} +\cfrac{t_{\gamma}^4}{n^2} + \cfrac{t_{\gamma}^2}{n} \cdot \left( \theta^* \right)^2}}{2 \cdot \left( 1 - \frac{t_{\gamma}^2}{n} \right)}  \right) $$

\vspace{4mm}

Підставивши значення отримаємо, що $ \theta \in (1.1792; 1.722) $ з ймовірністю $0.95$


\section{Висновки.}


Під час виконання данної розрахункової роботи було проведено первинну обробку деякої реалізації вибірки: було побудовано дискретний варіаційний ряд, полігон відносних частот, а також емпіричну функцію розподілу. Було знайдено значення незміщених оцінок математичного сподівання та дисперсії. Було обчислено значення деяких вибіркових характеристик генеральної сукупності, а саме вибіркову моду, вибіркову медіану, а також вибіркову асиметрію. Були порівняні полігони відносних частостей і полігони розподілу закону Паскаля при певних параметрах, а також графіки емпіричної функції розподілу і функції розподілу закону Паскаля. Згодом була висута гіпотеза про те, що генеральна сукупність, якою була отримана реалізація вибірки розподілена за законом Паскаля. Була знайдена точкова оцінка невідомого параметра, а також доведена її незміщеність, конзистнентність, ефективність і асимптотична нормальність. За допомогою криткрію $\chi^2$(Пірсона) було показано, що на рівні значущості 0.05 дані не суперечать висунутій гіпотезі про те, що генеральна сукупність розподілена за законом Паскаля з параметром $\theta = 1.41$. В кінці було отримано довірчий інтервал з рівнем надійності $\gamma = 0.95$ для параметра $\theta$, а також обчислено його межі для нашої реалізації вибірки.


\newpage{}

\section*{Додаток}

\hypertarget{d1}{[1]} - Каніовська І.Ю. Конспект з теорії ймовірностей та математичної статистики  ст. 93


\end{document}








%Якщо $\zeta \sim Pas\left( \theta \right)$, то $\mathbb{E} \zeta = \theta, \mathbb{D} \zeta = \theta + \theta^2$.






